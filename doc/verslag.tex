\documentclass[10pt,fleqn,a4paper]{article}
\usepackage[english]{babel}
\usepackage{graphicx} % plaatjes zijn goed
\usepackage{comment}  % uitcommentarieren van blokken oud spul
\usepackage[hmargin=2.5cm,vmargin=2cm]{geometry} % meer ruimte
\usepackage{amsmath,amssymb,latexsym}
\usepackage{float}

\newcommand{\keywords}[1]{\par\addvspace\baselineskip\noindent\enspace\ignorespaces{\textbf{Keywords:~}}#1}
\newcommand{\negmuchspace}{\negthinspace\negthinspace\negthinspace\negthinspace\negthinspace\negthinspace\negthinspace\negthinspace\negthinspace\negthinspace\negthinspace\negthinspace\negthinspace\negthinspace\negthinspace\negthinspace\negthinspace\negthinspace}
\newcommand{\ds}[1]{d^2_{#1}}
\newcommand{\dsa}[1]{\overline{d^2}_{#1}}

%opening
\title{Domain-Specific Questionnaires}
\author{Marten Veldthuis}
\addtolength{\parskip}{0.5\baselineskip}
\allowdisplaybreaks[1]

\begin{document}

\sloppy
\begin{twocolumn}
\twocolumn[
\maketitle
\begin{@twocolumnfalse}
\begin{abstract}
This paper has an abstract. But I will not write the abstract until
the paper is largely done. Makes sense, no?  

\keywords{Domain Specific Languages, Language Design, Ruby, Rails,
  Dynamic Programming, Questionnaires, Psychiatrics, Web-Applications}

\end{abstract}
\vspace{5mm}
\end{@twocolumnfalse}
]

\section{Introduction}

The Rob Giel Onderzoekscentrum (RGOc) is a research center developing
a product called RoQua. RoQua is a web application which allows
psychiatric departments of health-care providers to monitor patients by
periodically letting them fill out questionnaires over the Internet.

Right now, RoQua uses a third-party web service called GlobalPark to
display, and have patients fill in, the actual questionnaires. While
this is a good separation of concerns for the application, in section
\ref{sec:problems} we will outline the problems with GlobalPark. In
section \ref{sec:approach} we explain our solutions to these problems,
and the rationale behind them.

\section{Problems}\label{sec:problems}

GlobalPark is a costly service, and given the nature of the data, the
RGOc would rather not be reliant on a service which is not under its
own control.

Another problem the RGOc is experiencing with the GlobalPark service
is a matter of usability. Defining questionnaires through the web
interface of GlobalPark is a slow process. The web interface for
editing questionnaires is slow, and gets increasingly slower when
there are more questions defined for a questionnaire.

It's also not possible to define multiple views for the same
questionnaire. In RoQua, this is needed because for most
questionnaires, both a patient-version as well as a bulk-input version
needs to be available. Currently, this means questionnaires have to be
defined twice, both in GlobalPark and in RoQua itself.

Lastly, the way GlobalPark sends back information to RoQua is by doing
an HTTP GET request back to RoQua. This opens up a potential CSRF
security hole. Additionally, this interface contains character
encoding bugs, and GlobalPark is largely unresponsive to bug reports.

All things considered, the RGOc would like to move away from
GlobalPark and switch to a replacement of their own.

\subsection{Goals}

The goals for the project, which was given the name Quby, and will be
referred to as such from here on, are then defined as:

\begin{enumerate}
\item Replace all functionality of GlobalPark, for as far as we use it
  currently.
\item \label{dsl} Make it easier to define questionnaires.
\item \label{restful} Make the API interface with RoQua simpler.
\end{enumerate}

\section{Method overview}



\section{Future Work}

\section{Conclusion}

%\bibliographystyle{plain}
%\bibliography{MusicSimMDS}
\end{twocolumn}

\end{document}
